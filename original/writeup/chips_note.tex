%% LyX 2.3.6.2 created this file.  For more info, see http://www.lyx.org/.
%% Do not edit unless you really know what you are doing.
\documentclass[english]{article}
\usepackage[T1]{fontenc}
\usepackage[latin9]{inputenc}
\usepackage{graphicx}

\makeatletter

%%%%%%%%%%%%%%%%%%%%%%%%%%%%%% LyX specific LaTeX commands.
%% Because html converters don't know tabularnewline
\providecommand{\tabularnewline}{\\}

%%%%%%%%%%%%%%%%%%%%%%%%%%%%%% User specified LaTeX commands.
\usepackage{url}

\@ifundefined{showcaptionsetup}{}{%
 \PassOptionsToPackage{caption=false}{subfig}}
\usepackage{subfig}
\makeatother

\usepackage{babel}
\begin{document}
\title{Estimating the value of a CHIP.\thanks{This note was prepared by a Ph.D. Economist, whose identity is not
disclosed in order to comply with the terms of their present full-time
employment. The author thanks Kyle Bateman, David Spencer, Christian
Vom Lehn, and the anonymous reviewer for their helpful comments. All
remaining errors are the author's own.} }
\date{\today}

\maketitle
This note presents a conceptual framework to quantify the baseline
value of the CHIP currency tied to the global average of one hour
of unskilled work. Econometric estimates using panel data from 89
world countries over the period 1992-2019 suggest the CHIP value of
\$2.53 per hour.

\section{Introduction}

The contemporary post-Bretton-Woods monetary system is increasingly
criticized for its excessive reliance on several large central banks,
which often pursue opaque and discretionary policies, poor handling
of inflation, and inability to embrace modern technologies, such as
cryptographic currencies. Responding to these challenges, Bateman
(2022) proposes a novel money system anchoring means of payment to
the value of time and using complementary currency nicknamed CHIP.\footnote{This acronym originally came from \textquotedblleft Credit Hour In
Pool.\textquotedblright{}} Bateman argues that this novel system will offer a number of important
improvements, including 
\begin{enumerate}
\item maintaining a consistent value over time, 
\item offering better value and convenience to consumers, 
\item exhibiting greater resistance to theft and unethical manipulation
by business or government, and 
\item being compatible with sound and sustainable economic principles, including
free will and choice. 
\end{enumerate}
For purposes of standardization, Bateman suggests indexing (or linking)
the value of a CHIP to one hour of basic or unskilled work. The purpose
of this research note is to quantify the baseline value of the CHIP.
The first section of the note uses the theory of economic growth to
conceptualize the CHIP value as a distortion-free value of unskilled
labor compensation at the global scale. The second section outlines
the empirical approach to apply this definition to the panel data
of 89 world countries over the period 1992-2019, collected from different
public sources. The third section describes econometric estimates,
including the main result of pinning the CHIP value to \$2.53 per
hour. Finally, the last section concludes and suggests several venues
for future work. 

\section{Theory}

This section presents a simple economic framework whose core assumptions
are based on a Solow-Swan growth modeling framework (Solow 1956, Swan,
1956, Acemoglu 2009). The model deviates from this framework by allowing
for labor skill and human capital differentiation (Hanushek and Kimko,
2000). We assume that each country is an open economy (i.e., there
are no barriers to labor and capital migration inside and outside
the country), producing and consuming a unique final good. The economy
is in the steady state equilibrium, so all decisions are solutions
to the static problem.

\subsection{Households (Labor Supply)}

The economy is inhabited by a large number of households (or economic
agents), which supply labor hours, $L$, inelastically; there is no
leisure. Within each country, agents have different labor skills,
ranking from unskilled to highly skilled labor. Let $a_{i}$ be the
labor skill level of a household type $i$ with $a_{1}=1$ be the
skill level of the highest skilled labor category. Countries also
differ in terms of average level of households' human capital (e.g.,
average educational attainment), which we denote by $h$. Under this
assumption, the total value of the efficient (i.e., skill- and human
capital- adjusted) labor in a country $j$ is given by:

\begin{equation}
L_{s,j}=h_{j}\sum_{i}^{N}a_{i}L_{i,j},
\end{equation}

where $N$ is the number of household types according to their skill
levels. 

\subsection{Firms (Labor Demand)}

We assume that all firms in the economy are small, operate in a perfectly
competitive environment, and have access to the same production function
for the final good. The supply side of the economy can then be described
by a representative firm and a production function. We assume that
economy's production function is the Cobb-Douglas function\footnote{The Cobb-Douglas production function is the most common example of
a production function used in macroeconomics (see, e.g., Acemoglu
2009). }: 

\begin{equation}
Y_{j}=K_{j}^{\alpha_{j}}L_{s,j}^{1-\alpha_{j}},
\end{equation}

where, $Y$ is the output of the good in the economy, $K$ is the
aggregate capital endowment in the country owned by households, and
$\alpha$ is the capital compensation share in the economy. Let $y=Y/L_{s}$
and $k=K/L_{s}$ be the amount of output and capital per unit of efficient
labor. Then the economy's production function per unit of efficient
labor becomes

\begin{equation}
y_{j}=k_{j}^{\alpha_{j}}.
\end{equation}


\subsection{Labor Market Equilibrium }

Because the economy's labor supply is inelastic, the labor market
equilibrium is determined by the labor demand side resulting from
firms' profit maximization problem:

\begin{equation}
\pi_{j}=K_{j}^{\alpha_{j}}L_{s,j}^{1-\alpha_{j}}-w_{s,j}L_{s,j},
\end{equation}

where $\pi$ is the firm's profit and $w$ is the unit cost of efficient
labor (or the wage rate per unit of efficient labor). A firm's profit
maximization requires

\begin{equation}
\frac{\partial\pi_{j}}{\partial L_{s,j}}=0\Longrightarrow\left(1-\alpha_{j}\right)K_{j}^{\alpha_{j}}L_{s,j}^{-\alpha_{j}}=\left(1-\alpha_{j}\right)k_{j}^{\alpha_{j}}=w_{s,j}=\bar{w_{s}}.
\end{equation}

Equation (5) is the standard result from the economic theory, which
implies that labor compensation should equal the value of the marginal
product of efficient labor. In the open economy equilibrium, wages
are equalized across countries (Samuelson, 1948), so all wages equal
the global marginal product of efficient labor, $\bar{w_{s}}.$

\subsection{The value of CHIPS for labor compensation\label{subsec:The-value-of}}

Let us define the distortion factor, $\theta_{j}\equiv\frac{\left(1-\alpha_{j}\right)k_{j}^{\alpha_{j}}}{w_{s,j}}$,
as the ratio of the marginal product of efficient labor to efficient
(i.e., skill-weighted) labor compensation in the economyUnder the
model assumptions, the distortion factor should be equal to one. In
the real world, model assumptions will unlikely hold given barriers
to labor and capital flows, regulatory distortions (e.g., minimum
wage), and other market failures (such as, e.g., market power). This
implies that the distortion factor will also differ from one. In practical
terms, the distortion factor implies how much each economy's labor
should be compensated (or taxed) to restore the labor market equilibrium
defined by equation (5). We can then define the distortion-free value
of labor compensation as the product of the distortion factor and
the observed efficient labor compensation in the economy:

\begin{equation}
w_{s,j}^{e}=\theta_{j}w_{s,j}.
\end{equation}

Combining equation (6) with the definition of a CHIP (Bateman, 2022),
we define the value of the CHIP as the distortion-free value of unskilled
labor compensation.

\section{Estimation Approach }

\subsection{Method}

We estimate the economies' production function using the following
empirical specification:

\begin{equation}
\ln y_{j,t}=\alpha_{j}\ln k_{j,t}+\epsilon_{j,t},
\end{equation}

where $\alpha_{j}$ is the vector of coefficients (or country fixed
effects) to be estimated. Equation (7) is estimated by the ordinary
least squares (OLS) with fixed effects estimation method and is implemented
in R statistical software using \emph{fixest} package, \emph{feols}
function.\footnote{see \url{https://www.rdocumentation.org/packages/fixest/versions/0.8.4/topics/feols}} 

\subsection{Data}

We rely on three open data sources to estimate the CHIPS value of
labor. The data on the labor input (measured as total weekly hours
by employed population in a country) and labor compensation (in USD)
come from the ILOSTAT \emph{Labour Force Statistics }and \emph{Wages
and Working Time Statistics }databases.\footnote{see \url{https://ilostat.ilo.org/data/}}
The ILOSTAT data differentiates labor input across nine International
Standard Classification of Occupations 2008 (ISCO-08) categories\footnote{see \url{www.ilo.org/public/english/bureau/stat/isco/}}:
(i) managers, (ii) professionals, (iii) technicians and associate
professionals, (iv) clerical support workers, (v) service and sales
workers, (vi) skilled agricultural, forestry, and fishery workers,
(vii) craft and related trades workers, (viii) plant and machine operators,
and assemblers, and (ix) elementary occupations.\footnote{We exclude the armed forces and occupations not elsewhere classified.
Some countries use older ISCO classifications (e.g., ISCO-68 and ISCO-88),
which are converted into ISCO-08 using the correspondence tables provided
by ISCO.} To construct the efficient labor input, we assume that managers are
the highest skill category. We calculate the skill-level of in each
remaining category as the ratio of wages in this category relative
to the managers. Figure \ref{fig:skill} shows the average skill levels
of each labor category across our data sample. We calculate the average
wage in each country as an average of wages across different types
of labor weighted by the total number of labor hours in each labor
category. 

\begin{figure}
\caption{Average skill level by labor skill category}
\label{fig:skill}

\includegraphics[scale=0.07]{skill}

\end{figure}

The data on output, capital input, and human capital index come from
the Penn World Tables (PWT) 10.0 database (Feenstra and Timmer, 2015).
For measures of output, we use (i) real GDP at constant national prices
(in a million of 2017 USD, variable\emph{ rgdpna}) and (ii) output-side
real GDP at current purchasing power parities (PPPs, in a million
2017 USD, variable\emph{ cgdpo}). For measures of capital input, we
use (i) capital stock at constant national prices (in a million of
2017 USD, variable\emph{ rnna}) and (ii) capital stock at current
PPPs (in a million 2017 USD, variable\emph{ cn}). For a measure of
human capital, we use the PWT human capital index (variable \emph{hc}),
based on years of schooling and returns to education. We distinguish
between output and capital stock estimates using market exchange rates
and PPPs because there is no consensus in the economics literature
about which measure more adequately represents the supply side of
the economy, especially in the poor countries. While PPPs better approximate
values of tradable homogenous goods, they are also biased due to their
poor ability to measure differences in quality across goods and services.
This may result in a mechanical overvaluation of consumption bundles
because the relative prices used for valuing the bundles differ from
the transacted prices (Dowrick and Akmal, 2005). As physical capital
input is poorly tradable on the secondary market due to significant
sunk costs and its quality is difficult to measure, our preferred
specification is using measures in constant national prices in millions
of 2017 USD.

Finally, we use the US GDP implicit price deflator data from the St.
Louis FRED database\footnote{see \url{https://fred.stlouisfed.org/series/USAGDPDEFAISMEI}}
to convert nominal U.S. dollar-denominated wages to their real values.
After removing extreme outliers and observations with obvious measurement
errors, the final dataset for estimating the marginal product of labor
is an unbalanced panel comprising 2165 observations and covering 98
world countries from 1970 to 2019. Combining the marginal product
of labor estimates' data with the wage data yields the final 451 observations
covering 89 countries from 1992-2019. 

\section{Results}

Figure \ref{fig:mpl_w} plots the estimated average marginal product
of efficient labor versus real wage across countries in our sample.
The solid line in Figure \ref{fig:mpl_w} s a 45-degree line along
which local labor markets show no distortions, and the country's marginal
product of efficient labor equals its real wage. If the data point
lies above the 45-degree line, the country's marginal product of efficient
labor is less than the real wage. That is, workers in these countries
are overpaid relative to market equilibrium. Conversely, if the data
point lies below the 45-degree line, the country's marginal product
of efficient labor exceeds real wage, and workers in these countries
are underpaid relative to market equilibrium. We see that in most
world countries, data points lie close to the 45-degree line, which
indicates their labor markets are relatively undistorted.\footnote{Several countries (Brunei, Italy, Luxembourg) are outliers with unexpectedly
large labor productivities, which are likely due to measurement errors.} However, we also see that real wage greatly exceeds the marginal
product of efficient labor in most OECD economies. This could be due
to market distortions from minimum wage regulations, greater bargaining
power of labor unions, or labor market segmentation leading to certain
skill shortages and excessive wages. This could also be due to unobserved
technological differences that could affect the marginal product of
labor estimates, such as the productivity of capital in the information
and communication services sector, which accounts for most of the
productivity improvements in developed economies (Jorgenson, Ho, and
Stiroh, 2008). Finally, estimates can be biased in developing countries
with a large informal sector, where many small, unregistered establishments
are missing from the ILO sample frame.\footnote{\url{https://ilostat.ilo.org/resources/concepts-and-definitions/description-wages-and-working-time-statistics/}}.

\begin{figure}
\caption{Estimated marginal product of efficient labor vs. real wage}
\label{fig:mpl_w}
\centering{}\includegraphics[scale=0.1]{mpl_wage_scatter}
\end{figure}

Table \ref{table:chipsf} shows summary statistics for calculated
distortion factors (DF) based on the formula in section \ref{subsec:The-value-of}
using the following definitions: 
\begin{itemize}
\item DF I (preferred specification): the ratio of estimated marginal product
of labor to real wage where labor is measured in total effective hours
worked, using measures of output and capital in constant national
prices in millions of 2017 USD. 
\item DF II (assuming $\alpha_{j}=0$ or marginal product of labor is simply
efficient labor): the ratio of the country's \textquoteleft efficient\textquoteright{}
wage (i.e., the wage weighted by productivity of each worker category,
where the productivity is measured by the relative wage of a given
worker category to managers) to its average wage in a given year. 
\item DF III (assuming $a_{i}=1$ or skill productivities are similar across
all labor categories): the ratio of estimated marginal product of
labor to real wage where labor is measured in total hours worked (i.e.,
a simple summation of labor hours across all labor categories), using
measures of output and capital in constant national prices in millions
of 2017 USD. 
\item DF IV (assuming PPP conversion): the ratio of estimated marginal product
of labor to real wage where labor is measured in total effective hours
worked, using measures of output and capital in current PPPs in millions
of 2017 USD. 
\end{itemize}
\begin{table}[!htbp]
\centering \renewcommand*{\arraystretch}{1.1}\caption{Summary Statistics for CHIPS conversion factors}
\label{table:chipsf}\resizebox{\textwidth}{!}{ %
\begin{tabular}{lrrrrrrr}
\hline 
Distortion Factor & N & Mean & Std. Dev. & Min & Pctl. 25 & Pctl. 75 & Max\tabularnewline
\hline 
DF I & 451 & 1.05 & 1.17 & 0.05 & 0.61 & 1.2 & 18.57\tabularnewline
DF II & 451 & 1.24 & 0.18 & 0.94 & 1.13 & 1.32 & 2.84\tabularnewline
DF III & 446 & 0.52 & 0.63 & 0.03 & 0.26 & 0.62 & 10.35\tabularnewline
DF IV & 451 & 1.08 & 1.09 & 0.05 & 0.61 & 1.24 & 15.89\tabularnewline
\hline 
\end{tabular}}
\end{table}

We see that, consistent with the evidence in Figure \ref{fig:mpl_w},
the average distortion factor using preferred specification is close
to 1, which indicates that labor markets are globally efficient. However,
there is a significant dispersion across countries, with the distribution
of the DF I factor exhibiting a long right tail (Figure \ref{fig:chipshist},
left panel). The average distortion factor assuming PPP conversion
(DF IV) is very similar to our preferred specification, which indicates
that using PPP rather than nominal exchange rate conversion has a
small impact on the estimated DF factor value. Alternative definitions
using more restrictive assumptions result in biased estimates of the
distortion factor. The DF II factor is, on average, greater than one
and has a much smaller dispersion than other measures. Assuming away
capital in a country's production thus \emph{overestimates} the value
of the marginal product of labor. The DF III factor is, on average,
less than one and twice smaller than our preferred estimate of the
DF factor. Assuming that skill productivities are similar across all
labor categories thus \emph{underestimates} the value of the marginal
product of labor.

\begin{figure}
\caption{Distribution of DF I factor and adjusted wages across countries}
\label{fig:chipshist}
\begin{centering}
\subfloat{
\begin{centering}
\includegraphics[scale=0.06]{hist_chips}\includegraphics[scale=0.06]{hist_chips_wage}
\par\end{centering}
}
\par\end{centering}
\end{figure}

We are now ready to answer the main question of this research note:
how large countries' unskilled labor (elementary occupations) wages
would be if market distortions were eliminated? Our preferred measure,
DF I, offers the average local value of labor compensation of \$2.5
per hour for unskilled labor (Figure \ref{fig:chipshist}, right panel),
whose values range between \$0.047 (Rwanda) to \$8.06 (Italy). Finally,
we need to obtain the global value of labor compensation (i.e., when
labor mobility barriers are eliminated). To do so, we calculate the
average value of labor compensation, weighted by each country's contribution
to global GDP. This gives us the final value of unskilled labor compensation
that underpins the CHIP index: \$2.53 per hour.

\section{Conclusions and Directions for Future Work}

This note sets an important milestone in quantifying the value of
a novel currency, the CHIP. Using a conceptual framework engrained
in the theory of economic growth, we define the baseline CHIP value
as a distortion-free value of unskilled labor compensation at the
global scale. Applying this framework to the panel data of 89 world
countries from 1992-2019, collected from different public sources,
yields econometric estimates of the CHIP value at \$2.53 per hour.
While this number may seem low in the context of developed economies
(estimated value is roughly eight times smaller than unskilled labor
compensation in the United States), it is not surprising if put into
the context of the global unskilled labor force, dominated by low
and middle-income countries. While reasonably accurate, estimates
in this note are only the first attempt to tackle the complex problem
of calculating the base value of the CHIP. As with all new concept
developments, several improvements can be made to obtain better estimates,
especially for developed economies where larger discrepancies between
actual unskilled wages and theoretical predictions have been found.
They include, but are not limited to (i) endogenizing labor supply
decisions in the context of labor-leisure choice across time (see,
e.g., Turnovsky, 2000), (ii) explicitly incorporating technological
differences, especially the value of information and communication
technology capital across countries (Jorgenson, 2005), and addressing
measurement errors affecting the quality of labor productivity and
earnings data related to regional disparities and prevalence of informal
economy.

\section*{References}

Acemoglu, Daron. (2009). \emph{Introduction to Modern Economic Growth.}
Princeton University Press, Princeton New Jersey.

Bateman Kyle. (2022). \emph{Got Choices. }\url{https://gotchoices.org/book/epub/gotchoices.epub}

Dowrick, Steve and Muhammad Akmal. (2005). Contradictory Trends in
Global Income Inequality: A Tale of Two Biases. \emph{Review of Income
and Wealth} 51(2): 201-229.

Feenstra, Robert C., Inklaar, Robert, and Timmer, Marcel P. (2015).
The Next Generation of the Penn World Table. \emph{American Economic
Review}, 105(10): 3150-3182, available for download at \url{www.ggdc.net/pwt}

Hanushek, Eric A., and Kimko, Dennis D. (2000). Schooling, Labor-force
Quality, and the Growth of Nations. \emph{American Economic Review},
90(5): 1184-1208.

Jorgenson, Dale W., Mun S. Ho, and Kevin J. Stiroh. (2008). A Retrospective
Look at the US Productivity Growth Resurgence. \emph{Journal of Economic
Perspectives} 22(1): 3-24.

Jorgenson, Dale W. (2005). Accounting for Growth in the Information
Age. Chapter 10 in \emph{Handbook of Economic Growth}, 1, 743-815,
Elsevier. 

Samuelson, Paul A. (1948). International Trade and the Equalisation
of Factor Prices. \emph{Economic Journal}, 58(230): 163-184.

Solow, Robert. M. (1956). A Contribution to the Theory of Economic
Growth. \emph{The Quarterly Journal of Economics}, 70(1): 65-94.

Swan, Trevor W. (1956). Economic Growth and Capital Accumulation.
\emph{Economic Record}, 32(2): 334-361.

Turnovsky, Stephen J. (2000). Fiscal Policy, Elastic Labor Supply,
and Endogenous Growth. \emph{Journal of Monetary Economics}, 45(1),
185-210.
\end{document}
